This is the action branch you may find\+: Instead of using a position server as in the main branch here it is replaced by an action server /action/\+Position.action In this way the user can stop the robot in a current position simply press 0 after has pressed 1 in order to generate a new random target and setting the velocity of the robot To run the simulation in Gazebo simply copy the following command in the ubuntu shall\+: ~\newline
 
\begin{DoxyCode}{0}
\DoxyCodeLine{rosrun rt2\_assignment1 sim.launch}
\end{DoxyCode}
 \begin{DoxyVerb}A new simulation in Coppelia is implemented.
To run the simulation follow this simply instructions:
\end{DoxyVerb}
 
\begin{DoxyCode}{0}
\DoxyCodeLine{roscore \&      \#in order to launch the ROS master}
\end{DoxyCode}
 \begin{DoxyVerb}Only after having installed CoppeliaSim move into the executable folder and launch it
\end{DoxyVerb}
 
\begin{DoxyCode}{0}
\DoxyCodeLine{./coppeliaSim.sh}
\end{DoxyCode}
 \begin{DoxyVerb}Please check that ROS is correctly setted before proceed

Then add the Coppelia scene that you can find in this branch:
coppeliaScene.ttt

To launch the simulation digit the following command
\end{DoxyVerb}



\begin{DoxyCode}{0}
\DoxyCodeLine{roslaunch rt2\_assignment1 simVrep.launch}
\end{DoxyCode}
 